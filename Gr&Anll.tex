\documentclass[12pt]{report}
%% Language and font encodings
%\usepackage[english]{babel}
%\usepackage[utf8x]{inputenc}
%\usepackage[T1]{fontenc}
%% Sets page size and margins
\usepackage[a4paper,top=2cm,bottom=5cm,left=2.5cm,right=2cm]{geometry}%,marginparwidth=1.75cm
%% Useful packages
\usepackage{amsmath}
\usepackage{amsfonts}
\usepackage{fancyhdr}
%\fancyhf{}
%\renewcommand{\headrulewidth}{0pt}
%\lhead{Universidad Nacional de Ingenier\'ia\\Facultad de Ciencias}
%\rfoot{ no page\vline\\un isomorfismo de un grupo en s\'i mismo}
%\usepackage[colorlinks=true, allcolors=blue]{hyperref}
%\title{Lista de ejercicios\\ Grupos y Anillos}
%\date{}
\pagestyle{empty}
\begin{document}
\begin{flushleft}
Universidad Nacional de Ingenier\'ia\\Facultad de Ciencias\\ \vspace{1cm}
\end{flushleft}
%\maketitle
\begin{center}
\textbf{\Large{Lista de ejercicios\\ Grupos y Anillos}}
\end{center}
\begin{enumerate}
\item Sea $G$ un grupo abeliano. Sea H el subconjunto de $G$ que consiste de la identidad $e$ junto con todos los elementos de orden $2$. Muestre que $H$ es un subgrupo de $G$.
\item Si $G$ es un grupo abeliano, el subgrupo de torsi\'on de $G$ es el conjunto formado por los elementos de orden finito de G $$\text{Tors}(G)=\left\lbrace x\in G\;:\;\exists n\in\mathbb{N}\;:\; x^n=e\right\rbrace$$ 
($e$ es el elemento neutro de $G$).
\begin{enumerate}
\item Muestre que $\text{Tors}(G)$ es un subgrupo de $G$
\item Exhiba elementos de torsi\'on de $E_n:y^2=x^3-n^2x$, donde $n$ es un entero positivo. ¿Puede estimar un elemento de orden infinito para alg\'un $E_n$? D\'e alguna evidencia al respecto.
\end{enumerate}
\item Muestre que todo homomorfismo de grupos $\phi:G\rightarrow G'$ donde $|G|$ es primo, debe ser el homomorfismo trivial o un monomorfismo.
\item Sea $G$ un grupo finito y supongamos que el automorfismo\footnote{isomorfismo de un grupo en s\'i mismo} $T$ lleva m\'as de tres cuartos de los elementos de $G$ en sus inversos. Pruebe que $T(x)=x^{-1}$ para todo $x\in G$.
\item Sea $G$ un grupo finito abeliano en el cual el n\'umero de soluciones en $G$ de la ecuaci\'on $x^n=e$ es a lo m\'as $n$ para cada entero positivo $n$. Pruebe que $G$ debe ser un grupo c\'iclico.
\item Sea $G$ un grupo tal que la intersecci\'on de todos sus subgrupos diferentes de $\left\lbrace e\right\rbrace$ es un subgrupo diferente de $\left\lbrace e\right\lbrace$. Pruebe que todo elemento en $G$ posee orden finito.
\item Denominamos $radical$ de un ideal $I$ al conjunto $$\sqrt I=\left\lbrace x\in\ A:\;\text{existe }x\in A;\; x^n\in I\right\rbrace.$$ 
\begin{enumerate}
\item Muestre que $\sqrt I $ es un ideal de A.
\item Determine el radical de un ideal $n\mathbb{Z} \text{ de } \mathbb{Z}.$
\item Muestre que $\sqrt I\cap\sqrt J=\sqrt{I\cap J}$, $\sqrt{I+J}=\sqrt{\sqrt I+\sqrt J}.$
\end{enumerate}
\item Muestre que, par imagen inversa por la aplicaci\'on natural de $A$ en $A/I$, los ideales de $A/I$ corresponden biyectivamente a los ideales de $A$ que contienen a $I$.
\item Demuestre la versi\'on general del teorema chino: Sean $A_1, \cdots,A_r$ ideales de $A$ primos dos a dos, i.e. $A_i+A_j=A$ para $i\neq j$. La aplicaci\'on $$\frac{A}{A_1\cdots A_r}\rightarrow\frac{A}{A_1}\times\cdots\times\frac{A}{A_r},$$ que aplica $x\;$ m\'od $A_1\cdots A_r$ sobre ($x\;$  m\'od $A_1$,$x\;$ m\'od $A_2$,...,$x\;$ m\'od $A_r$) es un isomorfismo de anillos.
\item Considere la sucesi\'on de Lucas en un anillo $A$, definida a partir de un elemento $a\in A$:  $(V_n)_{n\neq 0}$ definida por $$V_0=2.1_A,\; V_1=a \;\;y\;\; V_{n+1}=aV_n-V_{n-1}, \;\;\;\;\;n\geq 1.$$ Demuestre la relaci\'on $$V_nV_m=V_{n+m}+V_{n-m}.$$
\item Sea $a\in A.$ ?`Bajo qu\'e condici\'on el polinomio $1-aT$ es invertible en el anillo $A[T]$?
\end{enumerate}
\begin{flushright}
O. Vel\'asquez
\end{flushright}
%\footnote{\vline un isomorfismo de un grupo en s\'i mismo}
\end{document}
